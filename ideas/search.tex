\documentclass[]{emulateapj}
\PassOptionsToPackage{hyphens}{url}\usepackage{hyperref}
\usepackage{natbib}
\usepackage{amssymb}
\usepackage{color}
\usepackage{amsmath,mathtools}
\usepackage{epsfig}
\usepackage[FIGTOPCAP]{subfigure}
\usepackage{afterpage}
\usepackage{enumerate}
\usepackage{multirow}
\usepackage{verbatim}
\usepackage{relsize}
\usepackage{tikz}
\usetikzlibrary{shapes.geometric, arrows}
\usepackage{morefloats}
\usepackage{wasysym}

\newcommand{\noop}[1]{}
\newcommand{\note}[1]{{\color{red} #1}}
\newcommand{\cn}{\note{(citation needed)\ }}
\newcommand{\unit}[1]{\ensuremath{\, \mathrm{#1}}}
\newcommand{\mearth}{\unit{M_\oplus}}
\newcommand{\rearth}{\unit{R_\oplus}}
\newcommand{\msun}{\unit{M_\odot}}
\newcommand{\lsun}{\unit{L_\odot}}
\newcommand{\mstar}{\unit{M_\star}}
\newcommand{\rj}{\ensuremath{R_\mathrm{J}}}
\newcommand{\avg}[1]{\langle #1 \rangle}
\newcommand{\bavg}[1]{\bigg\langle #1 \bigg\rangle}
\newcommand{\tab}[1]{\hspace{.2\textwidth}\rlap{#1}}
\DeclareMathOperator*{\argmin}{arg\,min}

\shorttitle{Transit Search}
\shortauthors{Luger 2016}

\begin{document}

\title{Transit Search}
\author{Rodrigo Luger}


\section{Transit search}
\label{sec:transit_search}
Defining
%
\begin{align}
\boldsymbol{\chi} \equiv \sum_n \lambda_n^2 \mathbf{X^2_n},
\end{align}
%
we may write
%
\begin{align}
\mathbf{m} = \left(\boldsymbol{\chi} + \lambda_\tau^2 \boldsymbol{\tau} \hspace{-2pt} \cdot \hspace{-2pt} \boldsymbol{\tau^\top} \right)
            \cdot
            \left(
            \boldsymbol{\chi} + \mathbf{K} + \lambda_\tau^2 \boldsymbol{\tau} \hspace{-2pt} \cdot \hspace{-2pt} \boldsymbol{\tau^\top}
            \right)^{-1} 
            \cdot
            \mathbf{y}.
\end{align}
Applying the Sherman-Morrison formula, the second term in parentheses may be expressed as
%
\begin{align}
\left( \boldsymbol{\chi} + \mathbf{K} \right)^{-1} 
-
\frac{
       \left( \boldsymbol{\chi} + \mathbf{K} \right)^{-1} 
       \cdot
       \boldsymbol{\tau} \hspace{-2pt} \cdot \hspace{-2pt} \boldsymbol{\tau^\top}
       \cdot
       \left( \boldsymbol{\chi} + \mathbf{K} \right)^{-1} 
}
{
       \lambda_\tau^{-2} + \boldsymbol{\tau^\top}
       \cdot
       \left( \boldsymbol{\chi} + \mathbf{K} \right)^{-1} 
       \cdot
       \boldsymbol{\tau}.
}
\end{align}
Defining
%
\begin{align}
\boldsymbol{\xi} \equiv \left( \boldsymbol{\chi} + \mathbf{K} \right)^{-1},
\end{align}
%
our model becomes
%
\begin{align}
\mathbf{m} = \left(\boldsymbol{\chi} + \lambda_\tau^2 \boldsymbol{\tau} \hspace{-2pt} \cdot \hspace{-2pt} \boldsymbol{\tau^\top} \right)
            \cdot
            \left(
            \boldsymbol{\xi} - \frac{\boldsymbol{\xi} \hspace{-2pt} \cdot \hspace{-2pt} \boldsymbol{\tau} \hspace{-2pt} \cdot \hspace{-2pt} \boldsymbol{\tau^\top} \hspace{-2pt} \cdot \hspace{-2pt} \boldsymbol{\xi}}
                             {\lambda_\tau^{-2} + \boldsymbol{\tau^\top} \hspace{-2pt} \cdot \hspace{-2pt} \boldsymbol{\xi} \hspace{-2pt} \cdot \hspace{-2pt} \boldsymbol{\tau}}
            \right)
            \cdot
            \mathbf{y}.
\end{align}

\end{document}